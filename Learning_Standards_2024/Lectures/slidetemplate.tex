\usetheme[progressbar=none,block=fill,sectionpage=none]{metropolis}

%\usepackage[T1]{fontenc}

\unitlength=1cm
\def\Put(#1,#2)#3{\leavevmode\makebox(0,0){\put(#1,#2){#3}}}

\usepackage{mdframed}
\usepackage{ifthen}
\usepackage{etoolbox}
\usepackage{pgfpages}
\usepackage{framed}

\usepackage{tabularx}

\newcolumntype{Y}{>{\centering\arraybackslash}X}

\mode<handout>{%
%    \pgfpagesuselayout{4 on 1}[letterpaper,border shrink=5mm,landscape]
%    \setbeameroption{show notes}
	\setbeameroption{show notes on second screen=right}
}

% These colours are good for colourblind people
% https://personal.sron.nl/~pault/
\definecolor{tolBlue}{HTML}{0077BB} 
\definecolor{tolCyan}{HTML}{33BBEE}
\definecolor{tolTeal}{HTML}{009988} 
\definecolor{tolOrange}{HTML}{EE7733} 
\definecolor{tolRed}{HTML}{CC3311} 
\definecolor{tolMagenta}{HTML}{EE3377} 
\definecolor{tolGrey}{HTML}{BBBBBB} 

\newenvironment{colourframe}[1][]{\begin{mdframed}[linewidth=2pt,linecolor=#1,backgroundcolor=black!2]}
  {\end{mdframed}}


\theoremstyle{definition}
\newtheorem{theorem}{Theorem}
\newtheorem{definition}[theorem]{Definition}
\newtheorem{defthm}[theorem]{Definition \& Theorem}
\newtheorem{example}[theorem]{Example}
\newtheorem{lem}[theorem]{Lemma}
\newtheorem{conjecture}[theorem]{Conjecture}
\newtheorem{construction}[theorem]{Construction}
\newtheorem{coroll}[theorem]{Corollary}
\newtheorem{remark}[theorem]{Remark}
\newtheorem{method}[theorem]{Method}
\newtheorem{recall}[theorem]{Recall}


\AtBeginEnvironment{definition}{%
	\setbeamercolor{block title}{fg=white,bg=tolTeal}
	\setbeamercolor{block body}{fg=black,bg=white!15}
}

\AtBeginEnvironment{defthm}{%
	\setbeamercolor{block title}{fg=white,bg=tolTeal}
	\setbeamercolor{block body}{fg=black,bg=tolTeal!15}
}

\AtBeginEnvironment{theorem}{%
	\setbeamercolor{block title}{fg=white,bg=tolOrange}
	\setbeamercolor{block body}{fg=black,bg=tolOrange!15}
}

\AtBeginEnvironment{method}{%
	\setbeamercolor{block title}{fg=white,bg=tolMagenta}
	\setbeamercolor{block body}{fg=black,bg=tolMagenta!15}
}

\newcommand{\printtitle}{
	\thispagestyle{empty}
	\begin{mdframed}[linewidth=0,backgroundcolor=tolBlue,fontcolor=white,font={\bfseries}]
		MAT188 Linear Algebra \hfill Fall 2023
	\end{mdframed}
	
	\vfill

	\begin{tikzpicture}
		\fill [tolOrange] (0,0) rectangle (1.5,1.5);
		\node [black!2] at (0.75,0.75) {\Huge\bfseries \themodulenumber};
		\node [anchor=west,text width=10cm] at (1.6,0.75) {PCE:\newline\Large\bfseries \themodule};
	\end{tikzpicture}
	
	\vfill

}


\newcommand{\smallA}{
	\begin{tikzpicture}[scale=0.5,baseline={([yshift=-0.7ex]current bounding box.center)}]
		\draw [ForestGreen,line width=2] (0,0) rectangle (0.8,0.8);
		\node at (0.4,0.4) {\bfseries A};
	\end{tikzpicture}\hspace{2mm}}

\newcommand{\smallB}{
	\begin{tikzpicture}[scale=0.5,baseline={([yshift=-0.7ex]current bounding box.center)}]
		\draw [RedOrange,line width=2] (4.5,0) rectangle (5.3,0.8);
		\node at (4.9,0.4) {\bfseries B};
	\end{tikzpicture}\hspace{2mm}}
	
\newcommand{\smallC}{
	\begin{tikzpicture}[scale=0.5,baseline={([yshift=-0.7ex]current bounding box.center)}]
		\draw [NavyBlue,line width=2] (9,0) rectangle (9.8,0.8);
		\node at (9.4,0.4) {\bfseries C};
	\end{tikzpicture}\hspace{2mm}}

\newcommand{\smallD}{	
	\begin{tikzpicture}[scale=0.5,baseline={([yshift=-0.7ex]current bounding box.center)}]
		\draw [Yellow,line width=2] (13.5,0) rectangle (14.3,0.8);
		\node at (13.9,0.4) {\bfseries D};
	\end{tikzpicture}\hspace{2mm}}
	
\newenvironment{teamwork}[1]{\begin{colourframe}[tolMagenta]\textbf{\color{tolMagenta}Teamwork {#1} min}

}{\end{colourframe}}

\newcounter{saveenumi}
\newcommand{\seti}{\setcounter{saveenumi}{\value{enumi}}}
\newcommand{\geti}{\setcounter{enumi}{\value{saveenumi}}}
\newenvironment{beanum}{\begin{enumerate}\geti}{\seti\end{enumerate}}

\newcommand{\framedfig}[1]{{%
\setlength{\fboxsep}{0pt}%
\setlength{\fboxrule}{1pt}%
\fbox{#1}%
}}

\newenvironment{amatrix}[1]{%
	\left(\begin{array}{@{}*{#1}{c}|c@{}}
	}{%
	\end{array}\right)
}



\newcommand{\ds}{\displaystyle}

\newcommand{\vocab}[1]{\textbf{#1}}

\newcommand{\RR}{\mathbb{R}}
\newcommand{\ZZ}{\mathbb{Z}}
\newcommand{\vect}[1]{\begin{bmatrix}#1\end{bmatrix}}
\newcommand{\matrixtwo}[1]{\begin{bmatrix}#1\end{bmatrix}}
\newcommand{\matrixthree}[1]{\left\lbrack\begin{array}{ccc}#1\end{array}\right\rbrack}
\newcommand{\matrixfour}[1]{\left\lbrack\begin{array}{cccc}#1\end{array}\right\rbrack}
\newcommand{\norm}[1]{\left\lVert#1\right\rVert}
\newcommand{\abs}[1]{\left\lvert#1\right\rvert}
\DeclareMathOperator{\Span}{Span}
\DeclareMathOperator{\Ker}{ker}
\DeclareMathOperator{\Range}{range}
\DeclareMathOperator{\Null}{null}
\DeclareMathOperator{\Dim}{dim}
\DeclareMathOperator{\Det}{det}
\DeclareMathOperator{\Row}{row}
\DeclareMathOperator{\Col}{col}
\DeclareMathOperator{\Rank}{rank}
\DeclareMathOperator{\Nullity}{nullity}
\DeclareMathOperator{\Adj}{adj}
\DeclareMathOperator{\Proj}{proj}
\DeclareMathOperator{\Curl}{curl}
\DeclareMathOperator{\Div}{div}
\newcommand{\uvecset}{\{u_1,\dots,u_m\}}
\newcommand{\unvecset}{\{u_1,\dots,u_n\}}
\newcommand{\suchthat}{\,:\,}
\newcommand{\ntm}{n \times m}
\newcommand{\ntn}{n \times n}
\newcommand{\Tmap}{T:\RR^m\to\RR^n}
\newcommand{\Tinv}{T^{-1}}
\newcommand{\Ainv}{A^{-1}}
\newcommand{\cmark}{\ding{51}}
\newcommand{\xmark}{\ding{55}}
\newcommand{\calA}{\mathcal{A}}
\newcommand{\calB}{\mathcal{B}}
\newcommand{\calU}{\mathcal{U}}
\newcommand{\calE}{\mathcal{E}}
\newcommand{\calS}{\mathcal{S}}
\newcommand{\limxyab}{\lim_{(x,y)\to(a,b)}}
\newcommand{\del}{\partial}
\newcommand{\dxdy}{\,dx\,dy}
\newcommand{\dydx}{\,dy\,dx}
\newcommand{\inv}{^{-1}}
\renewcommand{\v}[1]{\lbrack{#1}\rbrack}
\newcommand{\ip}[1]{\left\langle {#1} \right\rangle}